\documentclass{beamer}
\usetheme{AnnArbor}
\usefonttheme[onlymath]{serif} 
\usepackage[spanish]{babel}
\usepackage{lmodern}
\usepackage[T1]{fontenc}
\usepackage[utf8]{inputenc}
\usepackage{amsmath}
\usepackage{amsfonts}
\usepackage{amssymb}
\usepackage{graphicx}
\usepackage{epstopdf}
\usepackage{hyperref}
%\usepackage{minted}
\usepackage{adjustbox}
\usepackage{algorithm,algorithmic}

\usepackage{tikz}
\usetikzlibrary{mindmap,trees}
\usepackage{verbatim}



\setbeamerfont{author in head/foot}{size={\fontsize{3.5pt}{5pt}\selectfont}}
\author{ David Cardozo\inst{1}}
% - Give the names in the same order as the appear in the paper.
% - Use the \inst{?} command only if the authors have different
%   affiliation.
\title{Candidatura de TICs }
\subtitle{Propuestas para el Consejo Estudiantil} % A subtitle is optional and this may be deleted
%\logo{\includegraphics[height=0.8cm]{universidaddelosandes.png}\vspace{220pt}} 
\logo{\includegraphics[height=1.5cm]{nvologo.jpg}}%\logo{\includegraphics[height=0.8cm]{universidaddelosandescolombia.png}
\institute[Universidad de los Andes]
{
	\inst{1}%
	Representante Matemáticas \\
	Consejo Estudiantil Uniandino
}
\date{\today} % - Either use conference name or its abbreviation.
\subject{PDF Information} % This is only inserted into the PDF information catalog. Can be left out. 
%\setbeamercovered{transparent}
%\setbeamertemplate{navigation symbols}{}

\begin{document}
\maketitle

\begin{frame}[fragile, shrink]
	\begin{tikzpicture}
	\path[mindmap,concept color=black,text=white]
	node[concept] {Comite de las TIC's }
	[clockwise from=0]
	child[concept color=green!50!black] {
		node[concept] {Practicos}
		[clockwise from=90]
		child { node[concept] {Sitio Web} }
		child { node[concept] {Apoyo a Redes Sociales (Comunicación)} }
		child { node[concept] {Centro de Documentación} }
		child { node[concept] {Capacitación} }
	}  
	child[concept color=blue] {
		node[concept] { Talleres }
		[clockwise from=-30]
		child { node[concept] {Desarollo de tecnologías} }
		child { node[concept] {Directores de Comite-Formato} }
	}
	child[concept color=red] { node[concept] {Comunicación} }
	child[concept color=orange] { node[concept] {Generación de analísis}
		[clockwise from=-40] 
		child { node[concept] {Provedor de analísis estadístico} }
	};
	\end{tikzpicture}
	
\end{frame}

\begin{frame}
	\frametitle{Perfil}
	\begin{block}{Perfil Academico}
		Estudiante de Matemáticas, Física con opción en Educación y Gobierno 
	\end{block}
	\begin{block}{Experiencia}
		Plataforma Moodle para curso de Precalculo, Monitor del proyecto $ \Sigma $, Asociado en CONECTA-TÉ, Desarrollo y evaluación del curso \emph{Herramientas Computacionales}
	\end{block}
	
	\begin{block}{Tecnicos}
		\LaTeX, R, iPython, Unix y Web development.
	\end{block}
\end{frame}



\end{document}